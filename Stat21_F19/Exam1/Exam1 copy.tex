% Don't touch this %%%%%%%%%%%%%%%%%%%%%%%%%%%%%%%%%%%%%%%%%%%
\documentclass[11pt]{article}
\usepackage{fullpage}
\usepackage[left=1in,top=1in,right=1in,bottom=1in,headheight=3ex,headsep=3ex]{geometry}
\usepackage{graphicx}
\usepackage{float}

\newcommand{\blankline}{\quad\pagebreak[2]}
%%%%%%%%%%%%%%%%%%%%%%%%%%%%%%%%%%%%%%%%%%%%%%%%%%%%%%%%%%%%%%

% Modify Course title, instructor name, semester here %%%%%%%%

\title{Stat 21: Statistical Methods 2}
\author{Suzanne Thornton}
\date{Fall, 2019}

%%%%%%%%%%%%%%%%%%%%%%%%%%%%%%%%%%%%%%%%%%%%%%%%%%%%%%%%%%%%%%

% Don't touch this %%%%%%%%%%%%%%%%%%%%%%%%%%%%%%%%%%%%%%%%%%%
\usepackage[sc]{mathpazo}
\linespread{1.05} % Palatino needs more leading (space between lines)
\usepackage[T1]{fontenc}
\usepackage[mmddyyyy]{datetime}% http://ctan.org/pkg/datetime
\usepackage{advdate}% http://ctan.org/pkg/advdate
\newdateformat{syldate}{\twodigit{\THEMONTH}/\twodigit{\THEDAY}}
\newsavebox{\MONDAY}\savebox{\MONDAY}{Mon}% Mon
\newcommand{\week}[1]{%
	%  \cleardate{mydate}% Clear date
	% \newdate{mydate}{\the\day}{\the\month}{\the\year}% Store date
	\paragraph*{\kern-2ex\quad #1, \syldate{\today} - \AdvanceDate[4]\syldate{\today}:}% Set heading  \quad #1
	%  \setbox1=\hbox{\shortdayofweekname{\getdateday{mydate}}{\getdatemonth{mydate}}{\getdateyear{mydate}}}%
	\ifdim\wd1=\wd\MONDAY
	\AdvanceDate[7]
	\else
	\AdvanceDate[7]
	\fi%
}
\usepackage{setspace}
\usepackage{multicol}
%\usepackage{indentfirst}
\usepackage{fancyhdr,lastpage}
\usepackage{url}
\pagestyle{fancy}
\usepackage{hyperref}
\usepackage{lastpage}
\usepackage{amsmath}
\usepackage{layout}

\lhead{}
\chead{}
%%%%%%%%%%%%%%%%%%%%%%%%%%%%%%%%%%%%%%%%%%%%%%%%%%%%%%%%%%%%%%

% Modify header here %%%%%%%%%%%%%%%%%%%%%%%%%%%%%%%%%%%%%%%%%
\rhead{\begin{flushleft} Exam 1 \\ Stat 21 \\ October 4, 2019\end{flushleft}}

%%%%%%%%%%%%%%%%%%%%%%%%%%%%%%%%%%%%%%%%%%%%%%%%%%%%%%%%%%%%%%
% Don't touch this %%%%%%%%%%%%%%%%%%%%%%%%%%%%%%%%%%%%%%%%%%%
\lfoot{}
\cfoot{\small \thepage/\pageref*{LastPage}}
\rfoot{}

\usepackage{array, xcolor}
\usepackage{color,hyperref}
\definecolor{clemsonorange}{HTML}{EA6A20}
\hypersetup{colorlinks,breaklinks,linkcolor=clemsonorange,urlcolor=clemsonorange,anchorcolor=clemsonorange,citecolor=black}

%-------------------------------------------------------------------------%

\newcommand{\bel}{\begin{eqnarray}\label}
\newcommand{\eel}{\end{eqnarray}}
\newcommand{\bes}{\begin{eqnarray*}}
\newcommand{\ees}{\end{eqnarray*}}
\newcommand{\bei}{\begin{itemize}}
\newcommand{\eei}{\end{itemize}}
\def\bess{\bes\small }

\def\benu{\begin{enumerate}}
\def\eenu{\end{enumerate}}

\def\real{{\mathbb{R}}}
\def\R{{\real}}

\def\Var{\hbox{\rm Var}}
\def\E{\hbox{\rm E}}			



\newcommand{\oldtheta}{\theta^{\hbox{\tiny old}}}
\newcommand{\oldhbeta}{\hbeta^{\hbox{\tiny old}}}
\newcommand{\oldPsi}{\Psi^{\hbox{\tiny old}}}
\newcommand{\oldSigma}{\Sigma^{\hbox{\tiny old}}}
\newcommand{\newtheta}{\theta^{\hbox{\tiny new}}}
\newcommand{\newhbeta}{\hbeta^{\hbox{\tiny new}}}
\newcommand{\newPsi}{\Psi^{\hbox{\tiny new}}}
\newcommand{\newSigma}{\Sigma^{\hbox{\tiny new}}}
\newcommand{\mletheta}{\theta^{\hbox{\tiny mle}}}
\newcommand{\mlehbeta}{\hbeta^{\hbox{\tiny mle}}}
\newcommand{\mlePsi}{\Psi^{\hbox{\tiny mle}}}
\newcommand{\mleSigma}{\Sigma^{\hbox{\tiny mle}}}
\newcommand{\nj}[2]{\langle #1,#2\rangle}
\newcommand{\aaa}{\h{a}}
\newcommand{\LR}{\mathrm{LR}}
\newcommand{\sT}{\mathcal{T}}
\newcommand{\h}{\boldsymbol}


\renewcommand{\thefootnote}{\fnsymbol{footnote}}

\def\bess{\bes\small }
\newcommand{\convas}{\stackrel{{\rm a.s}}{\longrightarrow}}
\newcommand{\convp}{\stackrel{{\rm P}}{\longrightarrow}}
\newcommand{\convd}{\stackrel{{\rm D}}{\longrightarrow}}
\def\toD{\convd}
\def\towLik{\buildrel{\hbox{\rm w-lik}}\over\longrightarrow}
\def\toLik{\buildrel{\hbox{\rm lik}}\over\longrightarrow}
\def\benu{\begin{enumerate}}
	\def\eenu{\end{enumerate}}
\def\argmax{\mathop{\rm arg\, max}}
\def\argmin{\mathop{\rm arg\, min}}
\def\real{{\mathbb{R}}}
\def\R{{\real}}
\def\Var{\hbox{\rm Var}}
\def\E{\hbox{\rm E}}						 
\def\eps{\epsilon}
\def\veps{\varepsilon}			 


%-------------------------------------------------------------------------%
\begin{document}
\begin{flushright} 
	Name: \underline{\hspace{5cm}} 
	\end{flushright} 


\section*{Academic integrity }
[INSERT CLAUSE ABOUT ACADEMIC INTEGRITY]
\begin{flushright}
Signature{} \underline{\hspace{5cm}}      
\end{flushright}

%-------------------------------------------------------------------------%
\section*{Instructions:} There are three parts to this exam. Any time R code is quoted, ``$>$" marks a line of input.
\bigskip

\subsection*{Formulas}
Linear model: $Y = \beta_0 + \beta_1 x + \epsilon$  or, equivalently, $E[Y] = \beta_0 + \beta_1 x$\\
In the model(s) above, if we assume that the mean of $\epsilon$ is $0$ and the variance of $\epsilon$ is some unknown $\sigma^2$, then this means that the mean of $Y$ is $\beta_0 + \beta_1x$ and the variance of $Y$ is $\sigma^2$.


Fitted model: $\hat{y}_{i} = \hat{\beta}_1 + \hat{beta}_1x_i $\\
In the fitted model above, we solve for the least squares estimates of the parameters using these equations: 
$\hat{\beta}_1 = \frac{\sum_{i=1}^n(x_i-\bar{x}(y_i-\bar{y}))}{\sum_{i=1}^n(x_i-\bar{x})}$ \\
$\hat{\beta}_0 = \bar{y} - \hat{\beta}_1\bar{x}$

Definition of residuals ($e$): $\hat{y}_{i}  \hat{\beta}_1 + \hat{beta}_1x_i = e_i$\\

Recall, the properties of expectation and variance: 


Sums of squares: $\sum_{i=1}^{n}(y_i - \hat{y}_i)^2$\\
$\sum_{i=1}^{n}(\hat{y}_i-\bar{y})^2$\\
$\sum_{i=1}^{n}(y_i - \bar{y})^2$\\
Property of the sums of squares: $SS_{tot} = SS_{reg} + SS_{res}$

$\hat{\sigma} = \sqrt{\frac{SS_{res}}{n-2}}$ \\
$R^2 = 1-\frac{SS_{res}}{SS_{tot}} = \frac{SS_{reg}}{SS_{tot}}$ 


\begin{center}
\underline{\hspace{4in}} 
\end{center}
%-------------------------------------------------------------------------%

\section{} Suppose that the observational units in a study are patients who entered the emergency room at French Hospital in the previous week.  For each of the following, indicate whether it is a categorical variable, a numerical variable, or not a variable with regard to these observational units. a) How long the patient waits to be seen by a medical professional b) Whether or not the patient has health insurance c) Day of the week on which the patient arrives d) Average wait time before the patient is seen by a medical professional e) Whether or not wait times tend to be longer on weekends than weekdays f) Total cost of the emergency room visit

These are fairly straightforward for most students, but some struggle with the ones that are not variables at all (d, e).


\section{} Consider transactions at the on-campus snack bar to be the observational units in a statistical study.  State a research question that involves a categorical variable and a numerical variable for these observational units.  Also clearly identify and classify the two variables.

I have found that this question is very challenging for students.  I now realize that they need lots of practice with coming up with their own research questions.  I have in mind answers such as: Do people who pay with cash take longer to serve, on average, compared to people who pay with a card?  The explanatory variable is whether the customer pays with cash or card, which is categorical and binary. The response variable is how long the transaction takes to complete, which is numerical.


\section{} Critically analyze this summary and identify problems with the study/statistical analysis:

\href{psychology today paper}{https://www.psychologytoday.com/us/blog/human-flourishing/201909/does-religious-upbringing-promote-generosity-or-not?amp\&\_\_twitter\_impression=true}



\section{} Summarize what this paper is about based on the information in the title; 
Estimation and hypothesis testing in regression in the presence of nonhomogeneous error variances
\href{statistics article}{https://www.tandfonline.com/doi/abs/10.1080/03610918308812299?journalCode=lssp20}


\section{} Ruin this comic for me: explain the punchline of this joke using your knowledge of different sources of variation in the data 
\href{}{https://xkcd.com/1725}


\section{} What if there was no random error in our observations of Y? How do we find the line of best fit in this case?


\section{} Diamond data: give different plots and ask for conclusions about: heteroscedacity, normality, linearity

Say instead of the size of the diamond measured in carats, we'd like to look at the size in grams (1 carat $=$ 0.2 grams). 

\section{}
\subsection{} For running and health data: Given R output identify and interpret $\hat{\sigma}$, $\hat{\beta}_1$, $R^2$ within the context of the problem (be specific)

\subsection{}
Do a new version of this question as it relates to the slope parameter:  \href{}{https://askgoodquestions.blog/2019/08/26/8-end-of-the-alphabet}

\subsection{} make up (another new) statistic for model fit or coefficient significance and describe what kind of values they'd expect for skewed/dependent/heteroscedatic data. 


%-------------------------------------------------------------------------%
\end{document}



















